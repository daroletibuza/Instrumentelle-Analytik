\section{Diskussion der Ergebnisse}
\label{sec:diskussion}

Beginnend mit den Versuchen zur Löslichkeit des Rohproduktes ließ sich feststellen, dass sich das Produkt nicht in Wasser, zum Teil in Ethanol und vollständig in Essigsäureethylester löste (siehe Tab. \ref{tab:löslichkeit}). Grund hierfür werden die unterschiedlichen Polaritäten der Lösemittel sein. Da das Produkt selbst, durch seine aromatische Struktur (siehe Abb. \ref{fig:mechanismus}) eher unpolar ist, neigt es dazu sich auch eher in unpolaren Lösemitteln zu lösen. Dies bestätigt sich mit der Tatsache, dass Essigsäureethylester das unpolarste der getesteten Lösemittel ist und sich dort eine gute Löslichkeit verzeichnen lässt. Ebenso ist die umgekehrte Begründung wirksam, dass im vergleichsweise, polarsten Lösungsmittel Wasser sich das unpolare Produkt am schlechtesten lösen lässt.

%\anmerkung{Lösemittelabhängigkeit für die Addition}
%?cyclohexan: aprotisches Lösemittel --> keine Abgabe von H+ --> keine Reaktion mit Br-
%?wasser/Ethanol: protisch LM --> Abgabe von H+ --> Reaktion zu HBr

Die Schmelzpunkte zwischen Fraktion 1 und 2 unterscheiden sich eindeutig \mbox{(siehe Tab. \ref{tab:schmelzpunkte})}. Die Tatsache, dass Fraktion 2 einen höheren Schmelzbereich besitzt, lässt darauf schließen, dass in dieser Fraktion Anteilig mehr Reinprodukt vorhanden ist, als in Fraktion 1. Begründen lässt sich dies dadurch, dass organische Verbindungen mit einem höheren Anteil an Halogenen generell höhere Schmelzpunkte besitzen. Diese Tatsache beruht auf den elektrostatischen Wechselwirkungen der Moleküle zum Beispiel in Form von Dipol-Dipol-Wechselwirkungen. Zudem wirken die höheren molaren Massen der Halogenatome auch auf eine höhere molare Masse der Moleküle und verstärken somit auch die \textsc{Van-der-Waals}-Kräfte zwischen den Molekülen. Je mehr dieser elektrostatischen Kräfte zwischen den Molekülen wirken, desto mehr Energie ist für die Überwindung dieser Kräfte nötig, um Phasenübergang zu ermöglichen. Diese benötigte Energie wird im höheren Schmelzpunkt deutlich.\\

An dieser Stelle wird das auftreten von Nebenprodukten in diesem Versuch diskutiert. Diese gelten als Verunreinigungen für das Hauptprodukt (1,2-Dibrom-1-phenylethan). In dieser Diskussion wird angenommen, dass in diesem Versuch neben der gewünschten Additionsreaktion ebenfalls eine Substitutionsreaktion auftritt. Diese kann auftreten, wenn die Reaktion bei hohen Temperaturen abläuft oder das Brom radikalisch durch Absorption von Photonenenergie zerfällt (Stichwort: radikalische Substitution). In Abb. \ref{fig:nebenprodukte} sind mögliche Nebenprodukte einer solchen Substitution dargestellt. \\
\bild{Reaktion zu Nebenprodukten einer radikalische Substitution (vereinfacht)}{nebenprodukte}{0.6}
\newpage

Die Nebenprodukte unter Abb. \ref{fig:nebenprodukte} besitzen alles samt Doppelbindungen. Möchte man die Existenz dieser Verbindungen im Produkt nachweisen eignet sich beispielsweise das \textsc{Bayer}-Reagenz. Dieses entfärbt sich und bildet einen braunen Niederschlag, wenn Doppelbindungen bzw. die Nebenprodukte vorhanden sind. Alternativ bietet sich aufgrund der verschiedenen chemisch-physikalischen Eigenschaften der Verbindungen auch eine Gaschromatografie an. Über verschiedene Retentionszeiten könnten diese dann voneinander unterschieden werden. Untersuchungen mittels Infrarotspektroskopie erlauben sogar das Unterscheiden zwischen dem cis- und dem trans-Isomer des 2-Brom-1-phenylethens.

Die Ausbeute von rund \SI{65}{\percent} Reinprodukt wird vermutlich hauptsächlich durch die Bildung von Nebenprodukten und Rückständen in den jeweiligen Messgeräten und Apparaturen zu erklären sein. 