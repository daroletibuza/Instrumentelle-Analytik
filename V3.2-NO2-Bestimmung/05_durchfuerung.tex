\section{Versuchsdurchführung}
\label{sec:durchfuerung}

\subsection*{Probenahme der Raumluft}
Der Versuch begann um 8:02 Uhr mit der 90-minütigen Probenahme von \ce{NO2} unter einem Abzug im Labor Hg/E/2/17. 
Ziel ist es mit Hilfe des Versuchsaufbaus \mbox{(siehe Abb. \ref{fig:versuchsaufbau})} \ce{NO2} aus der Raumluft in \SI{25}{\milli \liter} \textsc{Saltzmann}-Lösung zu absorbieren.
Hierfür wurde nach Aufbau des Versuchsstandes die Pumpe eingeschaltet. Der Volumenstrom wurde am Ende der gesamten Versuchsdurchführung bestimmt.

\bild{Versuchsaufbau-Probenahme}{versuchsaufbau}{0.75}

\subsection*{Kalibrierung mit Natriumnitrit-Lösung}
Während die Probenahme lief wurden in der Zeit die Kalibrierlösungen hergestellt. Hierfür wurde eine Vergleichslösung mit einer Massenkonzentration von \SI{1,5}{\milli \gram \per \liter} Natriumnitrit zur Verfügung gestellt. Umgerechnet hatte die Lösung eine Massenkonzentration von \SI{1}{\milli \gram \per \liter} \ce{NO2}. In Tabelle \ref{tab:kalibrierlosungen} sind die Verdünnungsreihen nach der Versuchsanleitung mit den benötigten Volumina dargestellt.
\vspace*{-5mm}
\begin{table}[h!]
	\renewcommand*{\arraystretch}{1.2}
	\centering
	\rowcolors{2}{gray!25}{white}
	\caption{Kalibrierlösungen}
	\label{tab:kalibrierlosungen}
	\resizebox{\textwidth}{!}{
		\begin{tabulary}{1.2\textwidth}{C|C|CC}
			\hline
			\textbf{Kalibrierlösung} & \textbf{Zielkonzentration} $\left[\si{\milli \gram \per \liter}\right]$ & \textbf{Volumen Vergleichslösung} $\left[\si{\milli \liter}\right]$& \textbf{Volumen \textsc{Saltzmann}-Lösung} $\left[\si{\milli \liter}\right]$\\
			\hline
			K1 & 0,01 & 0,5 & 49,5\\
			K2 & 0,02 & 1,0 & 49,0\\
			K3 & 0,03 & 1,5 & 48,5\\
			K4 & 0,04 & 2,0 & 48,0\\
			K5 & 0,06 & 3,0 & 47,0\\
			K6 & 0,08 & 4,0 & 46,0\\	
			\hline			
	\end{tabulary}}
\end{table}%
\FloatBarrier
Je höher die Konzentration der Kalibrierlösung gewesen war, desto intensiver erschien die Farbe des Farbstoffes.
Nach dem Herstellen der Lösungen und 15 minütigem Warten wurden mit den Kalibrierlösungen K2, K4 und K5 die Wellenlänge des Absorbtionsmaximums $\lambda_{max}$ bestimmt. Zunächst ist dafür eine Küvette mit destilliertem Wasser als Referenz im Spektralphotometer vermessen und hinterlegt worden. Danach erfolgte die Vermessung der genannten Kalibrierlösungen und deren Messwerte für $\lambda_{max}$ wurden arithmetisch gemittelt. Die Ergebnisse dieser Messungen finden sich unter Abschnitt \ref{sec:ergebnisse}.

Nach der Bestimmung der Wellenlänge des Absorbtionsmaximums $\lambda_{max}$ konnten nun die Absorbanzen für alle Kalibrierlösungen bei dieser Wellenlänge bestimmt werden. Mit Hilfe dieser Absorbanzen ist nun ein Aufstellen der Kalibriergerade zur Messung der Konzentration der Probe möglich. Mehr dazu unter Abschnitt \ref{sec:ergebnisse}.

\subsection*{Messung der Raumluftprobe:}
Sobald die 90 Minuten vergangen waren wurde die Probenlösung nochmals für 15 Minuten stehen gelassen, sodass sich der Farbstoff vollständig ausbilden kann. Währenddessen wurde die Probenahmeapparatur abgebaut und die Pumpe an den Seifenblasenzähler angeschlossen.
Die Messung der Absorbanz der Raumluftprobe erfolgte nach Ablauf der Wartezeit genauso wie die Messung der Kalibrierlösungen. Es wurden drei Messungen von Absorbanzen bei der ermittelten Wellenlänge $\lambda_{max}$ für die Raumluftprobe durchgeführt.

\subsection*{Volumenstrom der Pumpe}
Der Volumenstrom der Pumpe wurde mittels Seifenblasenzähler ermittelt. Eine Skizze des Versuchsaufbaus ist in Abbildung \ref{fig:seifenblase} zu sehen. Hierfür wurde die Pumpe mit der oberen Schlauchtülle des Seifenblasenzählers verbunden und eingeschaltet. Am unteren Ende des Seifenblasenzählers wurde die Seifenblasenlösungen an die Öffnung gegeben, sodass diese von der Pumpe angesaugt wurde. Es bildeten sich flache Seifenblasen, welche sich entlang der Skalierung bis zum Doppelboden des Seifenblasenzähler bewegten. Nach dem es mehrere Blasen bis zum oberen Ende des Zählers geschafft hatten, wurde mit der eigentlichen Messung des Volumenstroms begonnen.\\
Hierfür wurde erneut eine Seifenblase durch ein Ansaugen der Pumpe im Seifenblasenzähler gebildet. 
\newpage

Sobald diese die beginnende Skalierung für den \SI{500}{\milli \liter}-Abschnitt des Zählers erreichte, wurde die Zeit gemessen die die Seifenblase brauchte, um die obere Marke von \SI{500}{\milli \liter} zu erreichen. Insgesamt wurde diese Messung dreimal durchgeführt und eine mittlere Zeit berechneten, die Seifenblasen benötigten. Aus diesem Wert wird unter \mbox{Abschnitt \ref{sec:ergebnisse}} der Volumenstrom der Pumpe bestimmt.

\bild{Skizze Seifenblasenzähler}{seifenblase}{0.33}
