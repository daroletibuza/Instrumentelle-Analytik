\section{Versuchsdurchführung}
\label{sec:durchfuerung}
Zunächst ist die Probe vorbereitet worden in dem ein paar Tropfen der Probe zusammen mit ein paar Tropfen des Lösungsmittels Deuterochloroform in das Proben Gefäß gegeben wurden. Das Probengefäß wurde in das Spektrometer eingesetzt und die weitere Versuchsdurchführung erfolgte am Computer. In diesem wurde innerhalb des Programms, spezifisch für das Spektrometer, das Experiment \texttt{1D Proton experiment} ausgewählt und die Probe entsprechend dem verwendeten Lösungsmittel gelockt. Weiterhin wurden zusammen mit einem Betreuer Tune und Match überprüft, ein Shimming der Probe erfolgte und danach die Parameter für die Impulsfolge überprüft. Nach diesen Bearbeitungsschritten wurde die Probe vermessen und durch das Programm mit Hilfe der Messsignale und der \textsc{Fourier}-Transformation ein Spektrum erzeugt. 

In die Tabelle \ref{tab:nmr_messparameter} sind die wichtigsten aufgenommenen Messparameter dargestellt.
\vspace*{-5mm}
\begin{table}[h!]
	\renewcommand*{\arraystretch}{1.2}
	\centering
	\rowcolors{2}{gray!25}{white}
	\caption{wichtigste Messparameter des NMR-Versuches}
	\label{tab:nmr_messparameter}
	\resizebox{0.75\textwidth}{!}{
		\begin{tabulary}{1.0\textwidth}{C|C|C}
			\hline
			\textbf{Messparameter} &\textbf{Symbol}&\textbf{Mittelwert}\\
			\hline
			Messfrequenz &\texttt{SF}&\SI{400,23}{\mega \hertz}\\
			Temperatur&\texttt{TE}&$\SI{303,2}{\kelvin}= \SI{30,05}{\celsius}$\\
			Lösungsmittel&\texttt{Solvent}&\ce{CDCl3} referenziert auf \SI{7,2}{\ppm}\\
			Pulslänge&\texttt{P1}&\SI{15}{\micro\second}\\
			Pulsdelay&\texttt{D1}&\SI{0}{\second}\\
			Anzahl der Scans&\texttt{NS}&1\\			
			\hline			
	\end{tabulary}}
\end{table}%
\FloatBarrier