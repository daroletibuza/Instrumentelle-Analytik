\section{Einleitung und Versuchsziel}
\label{sec:aufgabenstellung}
%In der Aufgabenstellung wird (in eigenen Worten und ganzen Sätzen) formuliert, was das Ziel des 
%Versuches ist.  
%[Beachten Sie die eigentliche Aufgabenstellung in den Versuchsanleitungen sowie die Hinweise zur Auswertung!] 

Im folgenden Versuch wird die Konzentration an Stickstoffdioxid in der Raumluft des Labors Hg/E/2/17 bestimmt. Da Stickstoffdioxid normalerweise nur in geringen Mengen emittiert wird, trifft man in diesem Versuch eine theoretische Annahme. Diese umfasst, dass im Labor beispielsweise ein Druckgefäß geplatzt ist oder der Abzug nicht ordnungsgemäß arbeitet und deshalb die Konzentration an \ce{NO2} in der Raumluft bestimmt werden muss.\\
Arbeitsmethodisch wird eine Langzeitbeprobung durchgeführt, um das \ce{NO2} anzureichern. Infolgedessen wird mittels externen Standards und der UV-Vis-Spektroskopie die Konzentration an \ce{NO2} bestimmt.
