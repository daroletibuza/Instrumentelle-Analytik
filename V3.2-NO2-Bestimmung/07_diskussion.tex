\section{Diskussion der Ergebnisse}
\label{sec:diskussion}
Grundlegend erscheinen die berechneten Raumluftkonzentrationen $\beta$ als plausibel, da mit sehr kleinen Konzentrationen zu rechnen war. Auch die Tatsache, dass damit der Arbeitsplatzgrenzwert unterschritten, ist erscheint sinnvoll, da keine offensichtliche Emissionsquelle im Praktikum vorbereitet oder ersichtlich gewesen war.\\
Die Kalibriergerade mit der die Konzentration an \ce{NO2} in der Absorptionslösung-Lösung bestimmt wurde, hat mit einem Bestimmtheitsmaß von $R^2=0,9992$ eine ausreichende Genauigkeit. Jedoch ist der Kalibrierbereich zu kritisieren. Alle Messungen der Raumluftprobe liegen unter dem kleinsten Wert der Kalibrierung. Zwar könnte man davon ausgehen, dass der Fehler in der Linearität gering ausfallen könnte, besser wäre jedoch wenn die kleinste Kalibrierlösung unter dem Messwert liegen würde. 
Da jedoch bereits mit dieser Genauigkeit eindeutig ist, dass der Messwert mit \SI{2,03}{\micro\gram \per \kmeter} unter dem Arbeitsplatzgrenzwert mit \SI{950}{\micro\gram \per \kmeter} liegt, ist eine weitere Anpassung des Verfahrens für diesen Verwendungszweck nicht sinnvoll.

Ist jedoch eine genauere Messung der \ce{NO2}-Konzentration nötig sollte die Kalibrierung entsprechend der zu erwartenden Messwerte angepasst werden.

Ebenfalls gilt es zu beachten, dass alle Messwerte dem Einfluss Messtoleranzen der genutzten Geräte sowie zufälligen Fehlern unterliegen.
