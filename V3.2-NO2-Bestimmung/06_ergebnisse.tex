\section{Ergebnisse}
\label{sec:ergebnisse}

\subsection*{Schmelzpunkt}
In Tabelle \ref{tab:schmelzpunkte} sind die Massen und Schmelzpunkte der Fraktionen 1 und 2 des Versuches aufgelistet. Auffallend ist, dass nach dem Umkristallisieren der Schmelzbereich des Produktes höher liegt als beim Rohprodukt.
\begin{table}[h!]
	\renewcommand*{\arraystretch}{1.2}
	\centering
	\rowcolors{2}{white}{gray!25}
	\caption{Massen und Schmelzpunkte der Produktfraktionen}
	\label{tab:schmelzpunkte}
	%\resizebox{10.5cm}{!}{
	\begin{tabulary}{1.0\textwidth}{C|CC}
		\hline
		\textbf{Fraktion} & \textbf{Fraktion 1: roh} & \textbf{Fraktion 2: rein}\\
		\textbf{Masse in $\left[\si{\gram}\right]$} & 20,90 & 0,88\\
		\textbf{Schmelzpunkt in $\left[\si{\celsius}\right]$} &67,5-71,2&70,2-72,6 \\
		\hline			
	\end{tabulary}
	%}
\end{table}%
\FloatBarrier

\pagebreak

\subsection*{Löslichkeitsprüfung}
In Tabelle \ref{tab:löslichkeit} ist dargestellt wie gut sich jeweils eine Spatelspitze des Rohproduktes in jeweils einem Milliliter Lösungsmittel (Wasser, Ethanol, Essigsäureethylester) gelöst haben. Der Versuch zeigte, dass sich das Rohprodukt am Besten im unpolarsten Lösungsmittel Essigsäureethylester lösen lies. 
\begin{table}[h!]
	\renewcommand*{\arraystretch}{1.2}
	\centering
	\rowcolors{2}{gray!25}{white}
	\caption{Löslichkeit des Rohproduktes in Wasser, Ethanol und Essigsäureethylester}
	\label{tab:löslichkeit}
	%\resizebox{10.5cm}{!}{
	\begin{tabulary}{1.0\textwidth}{C|C|C|C}
		\hline
		\textbf{Lösemittel} & \textbf{Wasser} & \textbf{Ethanol} & \textbf{Essigsäureethylester} \\
		\hline
		\textbf{Rohprodukt} & nicht löslich & schwer löslich & löslich\\
		\hline			
	\end{tabulary}
	%}
\end{table}%
\FloatBarrier

\subsection*{Ausbeute}
Die zu Beginn des Versuches abgemessenen \SI{11,5}{\milli \liter} frisch destilliertes Styrol und \SI{5,0}{\milli\liter} Brom entsprachen Stoffmengen von jeweils \SI{0,1}{\mol}. Die Massen der jeweiligen Produktfraktionen wurden eingewogen und sind in Tabelle \ref{tab:massen} aufgeführt. Molare Massen möglicher Nebenprodukte (siehe Abb. \ref{fig:nebenprodukte}) und des Hauptproduktes (1,2-Dibrom-1-phenylethan) sind mit \SI{182}{\gram \per \mole} und \SI{262}{\gram \per \mol} bestimmt worden.

\begin{table}[h!]
	\renewcommand*{\arraystretch}{1.2}
	\centering
	\rowcolors{2}{white}{gray!25}
	\caption{Massen der Produktfraktionen für die Bromierung von Styrol}
	\label{tab:massen}
	%\resizebox{10.5cm}{!}{
		\begin{tabulary}{1.0\textwidth}{C|CC|C}
			\hline
			\textbf{Fraktion} & \textbf{Masse} & \textbf{Stoffmenge} & \textbf{Notiz}\\
			\hline
			Ansatz& - &\SI{0,1}{\mol}&-\\
			Produktfraktion 1 (roh)& 20,9& &\\
			Produktfraktion 2 (rein)&0,88 & &\\
			Summe & 21,78& & Summe aus Fraktion 1+2\\
			\hline			
	\end{tabulary}
	%}
\end{table}%
\FloatBarrier

Aus diesen Angaben ergeben sich die folgenden Berechnungen zur \\ Ausbeuteberechnung:

\subsection*{Berechnung des Reinheitsgrades}
\textit{Verhältnis der Massen zwischen gereinigten Rohprodukt und Rohprodukt}
\begin{flalign}
	\eta_V	&= \frac{m_P(\SI{1,12}{\gram}) }{m_{RP}(\SI{1,12}{\gram})}\\
	&= \frac{\SI{0,88}{\gram}}{\SI{1,12}{\gram}} = \underline{\SI{78,6}{\percent}}
\end{flalign}

\subsection*{Umsatzgrad}
\textit{Umsatzgrad der Ausgangsstoffe zu Rohprodukt}
\begin{flalign}
	\eta &= \frac{m_{RP}*\eta_V}{n_{Styrol}*M_{\text{1,2-Dibrom-1-phenylethan}}} = \frac{\SI{21,78}{\gram}*\SI{78,6}{\percent}}{\SI{0,1}{\mole}*\SI{263,96}{\gram \per \mol}} = \underline{\SI{64,9}{\percent}}
\end{flalign}


%\anmerkung{Tabelle: Was wurde alles ausgewogen}
%Start: 0,1 mol
%Fraktion1: Reaktion 1 (Molare Masse HP oder NP benutzen)
%Fraktion2: Reaktion 2
%Rohprodukt(Fraktion 1+2): Ausbeute was generell umgesetzt wurde
%Produkt (aus 1g Rohprodukt): Ausbeute was durch Umkristallisation übrig bleibt --> Hochrechnung auf Rohprodukt


