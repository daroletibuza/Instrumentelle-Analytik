\newpage
\section{Versuchsdurchführung}
\label{sec:durchfuerung}

\bild{Versuchsaufbau}{aufbau}{0.6}

\subsection*{Durchführung:}
Zu Beginn wurde in einem \SI{250}{\milli \liter} Einhalskolben mit Magnetrührer \SI{11,5}{\milli \liter} Styrol in \SI{100}{\milli \liter} Cyclohexan vorgelegt. Die farblose Mischung wurde \SI{10}{\minute} unter Rühren bei \SI{400}{\rpm} in einem 8-\SI{10}{\celsius} kalten Wasserbad gekühlt. Danach ist eine Mischung aus \SI{50}{\milli \liter} Cyclohexan und \SI{5}{\milli \liter} Brom dazu getropft worden. Diese erschien zunächst braun bis blutrot. Durch Hinzutropfen des Broms in den Einhalskolben ist eine Reaktion zu beobachten gewesen. Diese äußert sich durch ein Verschwinden des rotbraunen Bromdampfes in sich das Brom an die Doppelbindung des Styrols anlagert. Das Zutropfen des Broms erfolgte über einen Zeitraum von \SI{1,5}{\hour} mit einer Wasserbadtemperatur zwischen 6-\SI{8}{\celsius}. Zuletzt wurde die Mischung weitere \SI{15}{\minute} beim Raumtemperatur gerührt und ein fein-kristalliner, gelber Feststoff fällt als Produkt aus.

\subsection*{Isolierung und Reinigung:}

Das ausgefallene Produkt wurde mit einem Büchner-Trichter abgesaugt. Der
Rückstand wurde ausgepresst und stellt die erste Produktfraktion dar. Für die
zweite Produktfraktion wurde aus dem Filtrat das Lösemittel mit Hilfe eines
Rotationsverdampfers bei verminderten Druck (235 mbar) abdestilliert.
Danach wurden \SI{1,12}{\gram} des Rohproduktes separiert und der Rest bis zum nächsten Praktikumsversuch im Abzug auf einem Tonteller getrocknet. Das Rohprodukt stellt in diesem Versuch Fraktion 1 dar.\\
Es folgte die Prüfung der Löslichkeit des Rohprodukts. Hierfür werden je \SI{500}{\milli \gram} des Rohproduktes in \SI{1}{\milli \liter} Lösungsmittel (Wasser, Ethanol, Essigsäureethylester) gelöst. Die Beobachtungen hierzu finden sich unter Abschnitt \ref{sec:ergebnisse}.

Zur Aufreinigung des Rohproduktes schloss sich der Löslichkeitsprüfung eine Umkristallisation des separierten Rohproduktes an. Hierfür wurde zunächst ein Gemisch von Ethanol und Wasser im Verhältnis 7:3 hergestellt. Für die eigentliche Umkristallisation wurden die \SI{1,12}{\gram} des Rohproduktes und \SI{7}{\milli \liter} des Ethanol-Wasser-Gemisches in einen \SI{50}{\milli \liter} Kolben gegeben. Der Inhalt des Kolbens wurde unter Rückfluss erhitzt bis sich der Kolbeninhalt gelöst hatte. Die Lösung wurde danach zuerst auf Raumtemperatur und dann im 8-\SI{10}{\celsius} kalten Wasser abgekühlt. Das auskristallisierte, gereinigte Produkt ist dann mit einem \textsc{Hirsch}-Trichter abgesaugt worden. Zum Schluss wurde mit kalten Ethanol-Wasser-Gemisch nachgespült. Das gereinigte Produkt stellt in diesem Versuch Fraktion 2 dar.

Fraktion 2 wurde ebenso wie Fraktion 1 bis zum nächsten Praktikumsversuch auf einem Tonteller unter einem Abzug getrocknet. Nach der Trocknung wurden die Fraktionen ausgewogen und die Schmelzpunkte bestimmt (siehe Abschnitt \ref{sec:ergebnisse}).\\
Beide Fraktionen sind danach zu einer Produktfraktion zusammengeführt worden.\\

\subsection*{Entsorgung}
Alle mit Brom verunreinigten Geräte wurden unter dem Abzug liegen gelassen, um das restliche Brom abdampfen zu lassen. Die wasserfreie Mutterlaugen, welche bei der Umkristallisation und den Löslichkeitsprüfungen entstanden, wurden im Behälter für halogenhaltige Abfälle entsorgt.
